\RequirePackage[l2tabu, orthodox]{nag}

\documentclass[a4paper]{article}

\usepackage[T1]{fontenc}
\usepackage[utf8]{inputenc}
\usepackage{lmodern}

\usepackage[english]{babel}
\usepackage{biblatex}

\usepackage{algpseudocode}
\usepackage{amssymb}
\usepackage{amsthm}
\usepackage{csquotes}
\usepackage{doi}
\usepackage{enumitem}
\usepackage{interval}
\usepackage{microtype}

\usepackage{hyperref}
\usepackage[subsection]{algorithm}

\addbibresource{provably-fair-algorithms.bib}

\newtheorem{proposition}[algorithm]{Proposition}
\theoremstyle{definition}
\newtheorem{definition}[algorithm]{Definition}
\theoremstyle{remark}
\newtheorem{remark}[algorithm]{Remark}
\newtheorem*{remark*}{Remark}

\newcommand{\concat}{\mathbin{\|}}

\begin{document}
\title{Introduction to Provably Fair Gaming Algorithms}
\author{Kristóf Poduszló \\ \href{mailto:kripod@protonmail.com}{kripod@protonmail.com}}
\date{September 1, 2017}
\maketitle

\begin{abstract}
As of today, the majority of the online gaming industry utilizes black box algorithms, forcing users to trust a third-party service for generating unbiased random data. Lately, a new paradigm has started to spread through the industry, paving the way towards transparent and verifiable algorithms being a standard in digital games.
\end{abstract}

\section{Introduction}
Provably fair algorithms bring a new era of opportunities to the online gaming industry. Unlike black box algorithms widely used amongst luck-based games involving stake \cite{gainsbury2013, riggedcasinos}, their fair counterparts are verifiable by anyone, including every participant of a particular game \cite{buterin2013}.

The algorithms described in this document represent the culmination of research about commitment schemes \cite{brassard1988, naor1991}, including, but not limited to, flipping a coin \cite{blum1983} \cite[pp.~243--245]{kraft2013} or playing a fair game of poker \cite{shamir1981, castellaroca2005} over the wire.

\section{Concepts}
\subsection{Verifiability of pseudorandom outputs}
Pseudorandom number generators provide a sequence of seemingly random outputs initialized by a seed. The presence of an initialization parameter provides the opportunity to use it as a key for verification of results.

A seed used in the algorithms covered by this document should consist of two main parts:

\begin{itemize}
\item \(hostSeed\): Shall be kept in secret until the end of a particular game. Similar to a private key in asymmetrical cryptography.

\item \(publicSeed\): Players should only generate or contribute to it (with equal amounts of influence) after a commitment (e.g.\ cryptographic hash) of \(hostSeed\) has been broadcast to every participant of a particular game.

\begin{remark}
Broadcasting a commitment of \(hostSeed\) amongst players not only protects \(hostSeed\) from being revealed early, but serves as a verification of integrity, proving that during a game, \(hostSeed\) cannot be tampered without notice.
\end{remark}
\end{itemize}

Using a mix of the entire \(hostSeed\) and \(publicSeed\) (e.g.\ by concatenating them) as an initialization parameter for randomization, every participant may have an influence on the outcome of results, with a negligible chance of manipulation\footnote{Given a commitment scheme which is computationally infeasible to break (e.g.\ based on a collision resistant cryptographic hash function).} in favor of any entity.

\begin{remark}
In a peer-to-peer network, every player is also a host, resulting in the presence of multiple \(hostSeeds\) and \(publicSeeds\) possibly paired to a \(privateKey\) and a corresponding \(publicKey\) for every participant. The \(publicKey\) of each player may also be used for commitments.
\end{remark}

\subsection{Initialization cycle}
A random \(hostSeed\) must be generated to initiate a new game.

\begin{enumerate}[label=(\alph*)]
\item Whether only a single player is betting against an operator, a \(hostSeed\) may be generated by the host using any source of entropy (preferably a true random number generator).

\item If multiple players are betting against an operator, a provably fair seeding event may be used to generate \(hostSeed\).

\begin{definition}
A provably fair seeding event \cite{seedingevent2015} makes it possible to generate \(publicSeed\) using a trustless randomization service (e.g.\ the hash of a specific upcoming block in the blockchain of a cryptocurrency), disallowing participants to have a direct influence on in-game randomization.
\end{definition}

\begin{remark}
When multiple players participate in a game, \(hostSeed\) shall not be generated by a single entity because that would allow a coalition to gain advantage over honest players by whispering \(hostSeed\) early to a selected group of participants.
\end{remark}

\item The problem of multiple players betting against each other may be solved by a mental poker protocol \cite{shamir1981}, which is beyond the scope of this document.
\end{enumerate}

Once \(hostSeed\) is revealed (optimally, at the end of a particular game), outputs generated by the algorithm become reproducible, proving that random results could not have been manipulated in favor of any entity.

\subsection{Definition and properties of a provably fair algorithm}
\begin{definition}
An algorithm behind a game is provably fair if and only if every player has the same amount of influence on in-game randomization in a verifiable manner.

\begin{remark}
Player-based randomization influence may be bypassed when substituted by an appropriate trustless seeding service.
\end{remark}
\end{definition}

\begin{proposition}{Necessary criteria of a provably fair algorithm}
\begin{enumerate}[label=(\roman*)]
\item Determinism (always produce the same output given a particular input).

\item A combination of the entire \(hostSeed\) and \(publicSeed\) is used for generating outputs (e.g.\ a keyed hash function\footnote{Unforgeability protects outputs from being predictable before \(hostSeed\) is revealed. For this purpose, the use of unkeyed hash functions or pseudorandom number generators is strongly discouraged.} like HMAC using \(hostSeed\) as key and \(publicSeed\) as message).

\item The integrity of \(hostSeed\) shall be verifiable by players through a commitment scheme (e.g.\ by publishing its cryptographic hash prior to the start of every particular game).

\item The algorithm must be public for every participant of the game.
\end{enumerate}
\end{proposition}

\section{Algorithms}
In this section, numerous generic fair algorithms will be proposed for games which are influenced by randomization, including, but not limited to, rolling a dice and shuffling a deck of cards.

\subsection{Generating a single random output}
The output generation function should be hard to invert \cite[pp.~30--35]{goldreich2007} in order to protect outputs from being predictable before \(hostSeed\) is revealed. While any entity in possession of \(hostSeed\) may predict the outputs of a provably fair algorithm, there should be no concern about fairness until every player has the same amount of information (preferably nothing) about \(hostSeed\) during a game.

\subsection{Generating a sequence of random outputs}

When multiple players participate in a game with numerous betting rounds following output generation, a new \(publicSeed\), influenced by every player or a trustless service, shall be used before each round in which bets may be placed.

In order to generate multiple outputs using a single set of seeds, a cryptographic \(nonce\) \cite[pp.~397--398]{menezes1996} should be utilized. A \(nonce\) used in provably fair algorithms shall be unique and predictable (e.g.\ it may represent the number of consecutive bets using the same \(hostSeed\), assuming the probability of a \(hostSeed\) collision is negligible).

A \(nonce\) may only be used once for a particular seed set, and shall be appended to the initial \(publicSeed\), producing a unique output for consecutive bets made using the same seeds.

Theoretically, an arbitrarily large output sequence can be generated using a bijective mathematical function \(\textrm{f}: \mathbb{N} \rightarrow \mathbb{R}\) (e.g.\ \(f(x) = x\)), agreed upon the initialization cycle of a game (until a commitment about \(hostSeed\) is made), as a \(nonce\) sequence provider.

Multiple parameters may be used to construct a \(nonce\) if necessary (e.g.\ when shuffling a deck of cards in a turn-based game, \(nonce\) should consist of both the round identifier and the shuffle state).

\section{Examples}
\begin{remark*}
Ensuring uniform distribution of random outputs is not in the scope of this document.
\end{remark*}

\subsection{Generating a single random integer}
The following functions generate a random integer based on a variant of the practically non-invertible HMAC (hash-based message authentication code) \cite{rfc2104} function using \(hostSeed\) as key and \(publicSeed\) as message.

\begin{algorithm}[H]
\caption{Generating a random integer in the range \(\interval[open right]{min}{max}\)}
\begin{algorithmic}
\Function{RandomInt}{$hostSeed, publicSeed, min, max$}
	\State \Return min + \((HMAC(hostSeed, publicSeed) \bmod (max - min))\)
\EndFunction
\end{algorithmic}
\end{algorithm}

\begin{algorithm}[H]
\caption{Rolling a dice}
\begin{algorithmic}
\Function{RollDice}{$hostSeed, publicSeed$}
	\State \Return \(RandomInt(hostSeed, publicSeed, 1, 6)\)
\EndFunction
\end{algorithmic}
\end{algorithm}

\subsection{Generating a sequence of random integers}
If multiple random outputs are required throughout a particular game, a \(nonce\) may be used to produce a sequence of random results. A \(nonce\) should be concatenated to \(publicSeed\) using a separator (e.g.\ \(":"\)).

\begin{algorithm}[H]
\caption{Shuffling an array (Fisher--Yates shuffle \cite{fisheryates1948, durstenfeld1964})}
\begin{algorithmic}
\Function{Shuffle}{$hostSeed, publicSeed, array$}
	\State \(n \gets array.length\)
	\For{\(i \gets 0, n - 2\)}
		\State \(j \gets RandomInt(hostSeed, publicSeed \concat ":" \concat i, i, n)\)
		\State \(Swap(array[i], array[j])\)
	\EndFor
\EndFunction
\end{algorithmic}
\end{algorithm}

\printbibliography
\end{document}
